
\documentclass{beamer}

\mode<presentation> {

%\usetheme{Antibes}
\usetheme{Luebeck}
%\usetheme{Malmoe}
%\usetheme{Darmstadt}

%\usetheme{Berlin}
%\usetheme{Copenhagen}
%\usetheme{Dresden}
%\usetheme{Frankfurt}
%\usetheme{Warsaw}

%\usecolortheme{albatross}
%\usecolortheme{beaver}
%\usecolortheme{beetle}
%\usecolortheme{crane}
%\usecolortheme{dolphin}
%\usecolortheme{dove}
%\usecolortheme{fly}
%\usecolortheme{lily}
%\usecolortheme{orchid}
%\usecolortheme{rose}
%\usecolortheme{seagull}
%\usecolortheme{seahorse}
%\usecolortheme{whale}
%\usecolortheme{wolverine}

%\setbeamertemplate{footline} % To remove the footer line in all slides uncomment this line
%\setbeamertemplate{footline}[page number] % To replace the footer line in all slides with a simple slide count uncomment this line

\setbeamertemplate{navigation symbols}{} % To remove the navigation symbols from the bottom of all slides uncomment this line
}

\usepackage{graphicx} % Allows including images
\usepackage{booktabs} % Allows the use of \toprule, \midrule and \bottomrule in tables
\usepackage{color, colortbl}



%----------------------------------------------------------------------------------------
%	TITLE PAGE
%----------------------------------------------------------------------------------------

\title[Scala Multisets]{Scala Multisets} % The short title appears at the bottom of every slide, the full title is only on the title page

\author{Nicolas Stucki} % Your name
\institute[UCLA] % Your institution as it will appear on the bottom of every slide, may be shorthand to save space
{
\'Ecole Polytechnique F\'ed\'erale de Lausanne (EPFL) \\ % Your institution for the title page
\medskip
\textit{nicolas.stucki@epfl.ch} % Your email address
}
\date{
26 January 2015\\[2em]
\includegraphics[height=0.7cm]{logos/epfl-logo.pdf} \hspace{1cm}
\includegraphics[height=0.7cm]{logos/scala-logo.pdf}


} % Date, can be changed to a custom date


\definecolor{beamer@blendedblue}{rgb}{0.8,0.0,0.0} % Red

\definecolor{scalared}{rgb}{0.8,0.0,0.0} % Red

%\setbeamercolor{alerted text}{fg=orange}
%\setbeamercolor{background canvas}{bg=white}
%\setbeamercolor{block body alerted}{bg=normal text.bg!90!black}
\setbeamercolor{block body}{bg=normal text.bg!95!black}
%\setbeamercolor{block body example}{bg=normal text.bg!90!black}
%\setbeamercolor{block title alerted}{use={normal text,alerted text},fg=alerted text.fg!75!normal text.fg,bg=normal text.bg!75!black}
\setbeamercolor{block title}{bg=scalared}
%\setbeamercolor{block title example}{use={normal text,example text},fg=example text.fg!75!normal text.fg,bg=normal text.bg!75!black}
%\setbeamercolor{fine separation line}{}
%\setbeamercolor{frametitle}{fg=brown}
%\setbeamercolor{item projected}{fg=black}
%\setbeamercolor{normal text}{bg=black,fg=yellow}
%\setbeamercolor{palette sidebar primary}{use=normal text,fg=normal text.fg}
%\setbeamercolor{palette sidebar quaternary}{use=structure,fg=structure.fg}
%\setbeamercolor{palette sidebar secondary}{use=structure,fg=structure.fg}
%\setbeamercolor{palette sidebar tertiary}{use=normal text,fg=normal text.fg}
%\setbeamercolor{section in sidebar}{fg=brown}
%\setbeamercolor{section in sidebar shaded}{fg= grey}
%\setbeamercolor{separation line}{}
%\setbeamercolor{sidebar}{bg=red}
%\setbeamercolor{sidebar}{parent=palette primary}
%\setbeamercolor{structure}{bg=black, fg=green}
%\setbeamercolor{subsection in sidebar}{fg=brown}
%\setbeamercolor{subsection in sidebar shaded}{fg= grey}
%\setbeamercolor{title}{fg=brown}
%\setbeamercolor{titlelike}{fg=brown}

\usepackage{listings}
% "define" Scala
\lstdefinelanguage{scala}{
  morekeywords={abstract,case,catch,class,def,%
    do,else,extends,false,final,finally,%
    for,if,implicit,import,macro,match,mixin,%
    new,null,object,override,package,%
    private,protected,requires,return,sealed,%
    specialized,super,this,throw,trait,true,try,%
    type,val,var,while,with,yield},
  otherkeywords={=>,<-,<\%,<:,>:,\#,@},
  sensitive=true,
  morecomment=[l]{//},
  morecomment=[n]{/*}{*/},
  morestring=[b]",
  morestring=[b]',
  morestring=[b]"""
}

\usepackage{color}
\definecolor{dkgreen}{rgb}{0,0.6,0}
\definecolor{gray}{rgb}{0.5,0.5,0.5}
\definecolor{mauve}{rgb}{0.58,0,0.82}
 
% Default settings for code listings
\lstset{frame=tb,
  language=scala,
  aboveskip=3mm,
  belowskip=3mm,
  showstringspaces=false,
  columns=flexible,
  basicstyle={\small\ttfamily},
  numbers=left,
%      numbersep=7pt,                   % how far the line-numbers are from the code
  numberstyle=\tiny\color{gray},
  keywordstyle=\color{scalared},
  commentstyle=\color{dkgreen},
  stringstyle=\color{mauve},
  frame=single,
  breaklines=true,
  breakatwhitespace=true
  tabsize=3
}

\begin{document}

\begin{frame}
\titlepage % Print the title page as the first slide
\end{frame}

\begin{frame}
\frametitle{Overview} % Table of contents slide, comment this block out to remove it
\tableofcontents % Throughout your presentation, if you choose to use \section{} and \subsection{} commands, these will automatically be printed on this slide as an overview of your presentation
\end{frame}

%----------------------------------------------------------------------------------------
%	PRESENTATION SLIDES
%----------------------------------------------------------------------------------------

%%%%%%%%%%%%%%%%%%%
\section{What are they?} 
%------------------------------------------------
\subsection{Multisets} 

\begin{frame}[fragile]
	\frametitle{Multisets in Mathematics}
	\begin{block}{Definition}
		A multiset (or bag), in mathematics, is a set like container where the elements may appear multiple times.
	\end{block}	
	\begin{block}{Multiset queries}
		\begin{itemize}
			\item $multiplicity$ : number of times some element is inside
			\item $\in$ : as in sets
			\item $\subseteq$ : multiset ``\emph{subset of} " (multiplicities are $\le$)
		\end{itemize}
	\end{block}
\end{frame}


\begin{frame}[fragile]
	\frametitle{Multisets in Mathematics}
	\begin{block}{Multiset operations}
		\begin{itemize}
			\item $\cup$ : Multiset union (agregation)
			\item $\uplus$ : Generalised set union (max. multiplicities)
			\item $\cap$ : Multiset intersection (min. multiplicities)
			\item $\setminus$ : Multiset minus (subtracting multiplicities)
		\end{itemize}
	\end{block}
\end{frame}

\begin{frame}[fragile]
	\frametitle{Multiset as Data Structures}
	\begin{block}{Objectives}
		\begin{itemize}
			\item Define a set like collection where elements can be repeated
			\item Faster multiset methods
			\item Decrease space used for dense multi sets
		\end{itemize}
	\end{block}
	\begin{block}{Found in a variety of languages}
		\begin{itemize}
			\item Java: Multisets of Guava library
			\item C/C++: std::multiset
			\item Python: Counter object
			\item Haskell: Data.MultiSet
		\end{itemize}
	\end{block}
\end{frame}


%------------------------------------------------
\subsection{Bag Trait}

\begin{frame}[fragile]
	\frametitle{\texttt{Bag} trait}
	\begin{block}{\texttt{Bag} trait}
		\begin{itemize}
			\item Trait that goes along \texttt{Seq}, \texttt{Set}, \texttt{Map} in the collection hierarchy
			\item Represents a Multiset 
			\item Elements grouped together when equivalent
			\item Flexible groups representation
		\end{itemize}
	\end{block}
\end{frame}

\begin{frame}[fragile]
	\frametitle{Bag trait}
	\begin{block}{Bag Properties}
		\begin{itemize}
			\item Sub-trait of \texttt{Iterable}
			\item Groups equivalent elements in buckets
			\item Allows any internal representation of buckets
		\end{itemize}
	\end{block}
	\begin{block}{Bucket (trait) Properties}
		\begin{itemize}
			\item Sub-trait of \texttt{Iterable}
			\item Contains only equivalent elements \\(== by default or any equiv)
			\item Has the same operations as Bag
		\end{itemize}
	\end{block}
\end{frame}

\begin{frame}[fragile]
	\frametitle{Bag trait}
	\begin{block}{Multiset operations}
		\begin{itemize}
			\item \texttt{union}: $\cup$
			\item \texttt{maxUnion}: $\uplus$
			\item \texttt{intersect}: $\cap$
			\item \texttt{diff}: $\setminus$
		\end{itemize}
	\end{block}
	\begin{block}{Lookup operations}
		\begin{itemize}
			\item \texttt{multiplicity(e)}: counts occurrences of element \texttt{e}
			\item \texttt{multiplicities}: returns a map of the multiplicities
		\end{itemize}
	\end{block}
\end{frame}

\begin{frame}[fragile]
	\frametitle{Bag trait}
	\begin{block}{Additions, removals and updates operations}
		\begin{itemize}
			\item Usual operators: \texttt{+}, \texttt{-}, \texttt{++}, ..., [\texttt{+=}, \texttt{-=}, ...]
			\item \texttt{added(e,n)} [\texttt{add(e,n)}]
			\item \texttt{removed(e,n)} [\texttt{remove(e,n)}]
			\item \texttt{withMultiplicity(e,n)} [\texttt{setMultiplicity(e,n)}]
		\end{itemize}
	\end{block}
	\begin{block}{Operations on equivalences (Bag only)}
		\begin{itemize}
			\item \texttt{apply(e)}: elements equivalent to \texttt{e} in a bag
			\item \texttt{mostCommon}: sub-bag with the most common elements
			\item \texttt{leastCommon}: sub-bag with the least common elements 
		\end{itemize}
	\end{block}
\end{frame}

\begin{frame}[fragile]
	\frametitle{Bag trait}
	\begin{block}{Iterators}
		\begin{itemize}
			\item \texttt{iterator}: normal iterator
			\item \texttt{distinctIterator}: iterator elements that are distinct
			\item \texttt{bucketsIterator}: iterator over buckets (Bag only)
		\end{itemize}
	\end{block}	
	\begin{block}{Representative element (Buckets only)}
		\begin{itemize}
			\item \texttt{sentinel}: element that is equivalent to all others in the bucket
		\end{itemize}
	\end{block}
\end{frame}


%%%%%%%%%%%%%%%%%%%
\section{Concrete implementations} 
%------------------------------------------------
\subsection{HashBag and TreeBag}

\begin{frame}[fragile]
	\frametitle{HashBag and TreeBag}
	\begin{block}{HashBag}
		\begin{itemize}
			\item Keeps buckets in a hash table
		\end{itemize}
	\end{block}	
	\begin{block}{TreeBag}
		\begin{itemize}
			\item Keeps buckets in a red-black tree
		\end{itemize}
	\end{block}	
	%\includegraphics[scale=0.4]{coll}
\end{frame}

\subsection{Example}

\begin{frame}[fragile]
	\frametitle{Example}
	\begin{lstlisting}
		// immutable Bag with a hash table containing the bag buckets
		import scala.collection.Bag
		import scala.collection.immutable.{HashBag=>Bag}
		// or immutable Bag with a red-black tree containing the buckets
		import scala.collection.immutable.{TreeBag=>Bag}
		// or mutable Bag with a hash table containing the bag buckets
		import scala.collection.mutable.{HashBag=>Bag}
		// or mutable Bag with a red-black tree containing the buckets
		import scala.collection.mutable.{TreeBag=>Bag}
	\end{lstlisting}
\end{frame}

\begin{frame}[fragile]
	\frametitle{Example}
	\begin{lstlisting}
		val bag = Bag.from("cat"->1, "dog"->3, "mouse"->1)
		// or 
		val bag = Bag("cat", "dog", "dog", "dog", "mouse")
		// or 
		val bag = Bag.empty[String] + "cat" + "dog" + "dog" + "dog" + "mouse"
		// or 
		val bag = Bag.empty[String] added ("cat"->1) added ("dog"->3) added ("mouse"->1)
		// or 
		val bag = Bag.empty[String] withMultiplicity ("cat", 1) withMultiplicity ("dog", 3) withMultiplicity ("mouse", 1)
		
		bag.toString // result:  Bag(dog, dog, dog; cat; mouse)
	\end{lstlisting}
\end{frame}

\begin{frame}[fragile]
	\frametitle{Example}
	\begin{lstlisting}
		val bag = Bag("cat", "dog", "dog", "dog", "mouse")
		val bag2 = Bag("cat", "mouse")

		bag("dog") // result:  Bag(dog, dog, dog)
		
		bag.multiplicity("cat")
		
		bag union bag2
		bag intersect bag2
		bag diff bag2
		bag maxUnion bag2
		
		for (animal <- bag) { ... }
	\end{lstlisting}
\end{frame}


%%%%%%%%%%%%%%%%%%%
\section{Bag configuration} 
\begin{frame}[fragile]
	\frametitle{Bag configuration}
	\begin{block}{Custom equivalences}
		\begin{itemize}
			\item Change the way elements are grouped together
		\end{itemize}
	\end{block}
	\begin{block}{Bucket representation}
		\begin{itemize}
			\item Change the way buckets are represented
		\end{itemize}
	\end{block}
	\begin{block}{\texttt{BagConfiguration} instance}
		\begin{itemize}
			\item Has one defined equivalence
			\item Generates instances of buckets
			\item Each \texttt{Bag} has one
		\end{itemize}
	\end{block}
\end{frame}

%------------------------------------------------
\subsection{Custom equivalence} 

\begin{frame}[fragile]
	\frametitle{Equivalences}
	\begin{block}{Equivalence relation}
		\begin{itemize}
			\item \emph{reflexive}: \texttt{equiv(x,x)==true}  for any \texttt{x}
			\item \emph{symmetric}: \texttt{equiv(x,y)==equiv(y,x)} for any \texttt{x} and any \texttt{y}
			\item \emph{transitive}: if \texttt{equiv(x,y)==true \&\& equiv(y,z)==true} then \texttt{equiv(x,z)==true}
		\end{itemize}
	\end{block}	
	\begin{block}{Scala \texttt{Equiv[T]} trait}
		\begin{itemize}
			\item \texttt{equiv(x:T,y:T):Boolean }
		\end{itemize}
	\end{block}	
\end{frame}

\begin{frame}[fragile]
	\frametitle{Ordered Equivalence}
	\begin{block}{Equivalence for TreeBag}
		\begin{itemize}
			\item \texttt{TreeBag} also require \texttt{Ordering}
			\item Use \texttt{Ordering} as equivalence
		\end{itemize}
	\end{block}
	\begin{lstlisting}
		val stringEquiv =   new Ordering[String] {
		   def compare(x: String, y: String): Int = x.size compare y.size
		}
	\end{lstlisting}
\end{frame}

\begin{frame}[fragile]
	\frametitle{Hashed Equivalence}
	\begin{block}{Equivalence for HashBag}
		\begin{itemize}
			\item \texttt{HashBag} also require a hash code
			\item Hash code must be coherent with equivalence
			\item Use \texttt{Equiv[T] with Hashing[T]} as equivalence
		\end{itemize}
	\end{block}
	\begin{lstlisting}
		val stringEquiv =   new Equiv[String] with Hashing[String] {
		   def equiv(x: String, y: String): Boolean = x.toLowerCase == y.toLowerCase
		   def hash(x: String): Int = x.toLowerCase.hashCode
		}
\end{lstlisting}
\end{frame}

%------------------------------------------------
\subsection{Bucket representation} 

\begin{frame}[fragile]
	\frametitle{Bucket representation}
	\begin{block}{\texttt{MultiplicityBucket}}
		\begin{itemize}
			\item 2-Tuple with reference to element and it's multiplicity
			\item Most compact representation
			\item Only on equality (no custom equivalence)
		\end{itemize}
	\end{block}
	\begin{block}{\texttt{BagOfMultiplicities}}
		\begin{itemize}
			\item Bag with \texttt{MultiplicityBucket}
			\item Compact representation
			\item Allows custom equivalence
		\end{itemize}
	\end{block}
\end{frame}

\begin{frame}[fragile]
	\frametitle{Bucket representation}
	\begin{block}{\texttt{ListBucket}}
		\begin{itemize}
			\item Keeps all elements in a \texttt{List}
			\item Full representation
			\item Doesn't lose references to objects
			\item Allows custom equivalence
		\end{itemize}
	\end{block}
	
	\begin{block}{Custom implementations}
		\begin{itemize}
			\item Uses can implements their own bucket
			\item Need to define a new \texttt{BagConfiguration}
		\end{itemize}
	\end{block}
\end{frame}

\subsection{Bag Configuration instances} 

\begin{frame}[fragile]
	\frametitle{Bag Configuration instances}
	\begin{block}{Bag configuration}
		\begin{itemize}
			\item Created with methods in \texttt{Bag.configuration}
			\item Set as implicit value
		\end{itemize}
	\end{block}
	\begin{lstlisting}
		implicit val conf = Bag.configuration.compact[T]
		implicit val conf = Bag.configuration.compactWithEquiv[T](equiv)
		implicit val conf = Bag.configuration.keepAll[T]
		implicit val conf = Bag.configuration.keepAll[T](equiv)
	\end{lstlisting}
\end{frame}


\subsection{Example} 

\begin{frame}[fragile]
	\frametitle{Example}

	\begin{lstlisting}
	object StringOrd extends Ordering[String] {
	   def compare(x: String, y: String): Int = x.size compare y.size
	}
	
	implicit val conf = TreeBag.configuration.compactWithEquiv(StringOrd)
	
	val bag = TreeBag("cat", "cat", "mouse","fish", "elephant", "dog", "bear" )
	
	bag.toString() 
	// result:  TreeBag(cat, cat, dog; bear, fish; mouse; elephant)
	\end{lstlisting}
\end{frame}


\begin{frame}[fragile]
	\frametitle{Example}

	\begin{lstlisting}
	// bag = TreeBag(cat, cat, dog; bear, fish; mouse; elephant)
	
	bag("cat") // result:  TreeBag("cat", "cat", "dog")
	bag("ooo") // result:  TreeBag("cat", "cat", "dog")
	bag("xxxx") // result:  TreeBag("fish", "bear")
	bag("") // result:  TreeBag()

	bag.mostCommon 
	// result:   TreeBag("cat", "cat", "dog") 
	
	bag.leastCommon 
	// result:  TreeBag("mouse"; "elephant") 
	\end{lstlisting}
\end{frame}


%------------------------------------------------
\section{} 
%------------------------------------------------

\begin{frame}
 \vspace{3cm}
\Huge{\centerline{Any questions?}} \vspace{2cm}
\centerline{\small{\url{https://github.com/nicolasstucki/multisets}}}
\end{frame}

%----------------------------------------------------------------------------------------

\end{document} 