I% Chapter Template

\chapter{Related and Future Work} % Main chapter title

\label{RelatedWork} % Change X to a consecutive number; for referencing this chapter elsewhere, use \ref{ChapterX}

\lhead{\emph{Related and Future Work}} % Change X to a consecutive number; this is for the header on each page - perhaps a shortened title

%----------------------------------------------------------------------------------------
%	SECTION Related Work
%----------------------------------------------------------------------------------------
\section{Related Work}
\newcommand{\itemspace}{\vspace{-1mm}}
List of related subjects:
\itemspace
\begin{itemize}
  \itemspace\item RRB Trees and Vectors \cite{RRBTrees, lorange2014rrb, cormen2001introduction, scalaVector211, scalaParVector211}
  \itemspace\item Scala \cite{odersky2008programming,36605}
  \itemspace\item Scala Collections \cite{ 1979992642, Oliveira:2010:TCO:1869459.1869489, EPFL-REPORT-200245, scalaVector211, FightingbitRotwithTypes, EPFL-CONF-170981}
  \itemspace\item Scala Parallel Collections \cite{collect11,Lea:2000:JFF:337449.337465, 6264/THESES, Prokopec:2014aa,scalaParVector211}
 \itemspace\item Functional data structures and semi-mutable data structures and related data structures \cite{Okasaki:1998:PFD:280586,DBLP:journals/jfp/HinzeP06, Driscoll:1989:MDS:64313.64317,4637966, Bayer:1970:OML:1734663.1734671, SPE:SPE4380251203, Bagwell01idealhash, 31/LAMP}
  \itemspace\item Performance and Code specialization \cite{5456/THESES, Chafi:2011:DAH:1941553.1941561, Coutts:2007:SFL:1291151.1291199, scalablitz15}
  \itemspace\item JVM: Arrays, GC, JIT compiler \cite{Kotzmann:2008:DJH:1369396.1370017, Paleczny:2001:JHT:1267847.1267848, Wurthinger:2011:EGC:2048147.2048168}
  \itemspace\item ScalaMeter \cite{Georges:2007:SRJ:1297027.1297033,scalameter}
  \itemspace\item Scala Test \cite{scalatest}
  \itemspace\item Scala Reflection and Quasiquotes \cite{EPFL-CONF-186844, quasiquotes}
  \itemspace\item Type specialization and Miniboxing \cite{Ureche:2013:MIS:2509136.2509537, Ureche:2014:LDL:2660193.2660197, Stucki:2013:BIS:2489837.2489847, 4820/THESES, EPFL-REPORT-200245}
\end{itemize}



%----------------------------------------------------------------------------------------
%	SECTION Related Work
%----------------------------------------------------------------------------------------
\section{Future Work}

\paragraph{Mesure unbalance}
% find a good measurement of unbalancess to characterise vectors
% characterize vectors on real world programs
One analysis that was left out of the scope of this project was the characterization of the vectors unbalance. There is currently no way to quantitatively measure the unbalance of on the tree node. Some ideas for this are: number of unbalanced nodes, number of balanced subtrees, average height of balanced subtrees, ... 

Once this measurement exists it would be possible to conduct a real world application characterization of these vectors. And see how common the unbalanced vectors are and if they are only slightly unbalanced or extremely unbalanced. From this it would be possible to give a true expected performance for the cases where the performances differ.

\paragraph{Miniboxing}
As discussed in section \ref{Arrays}, when primitive types are used in vector they become boxed. A proposed solution to this phenomenon is using Miniboxing \cite{Ureche:2013:MIS:2509136.2509537,  EPFL-REPORT-200245} on the vectors. Applying this transformation on the type parameter of the vector will remove boxing in the method calls. Additionally, this allows the use of \texttt{MbArray} on the leafs, these arrays take advantage of the additional type information that Miniboxing provides and applies it opportunistically. A prototype of this vector is already implemented in this project repository\footnote{Project \cite{projecRepo} under the \texttt{adding-miniboxing} branch.}. The prototype shows that the boxing is removed and that the leafs arrays are getting specialized. Tests and benchmarks specific to this implementation details are still required.


\paragraph{Simplify Code}
% use macros to define core operations of the vector to simplify expanded code (half of the work is already done on the generators)
It should be possible to simplify and reduce the amount of code of the Scala implementation of vector using Scala Macros \cite{EPFL-CONF-186844}  to expand and optimize the code that was done manually. This was partially done on the vector implementation generators (see \ref{ImplementationGenerators}), where some key abstractions generate most of the expansions. This is a good basis to write such macros.

% fallthroug switch
Another way to simplify the code would involve the creation of a new abstraction that would compile down to bytecode \texttt{tableswitch} instructions with fallthroughs. A switch with no \texttt{break}s should be enough\footnote{Look at \texttt{copyDisplays} function in the code, the \texttt{if/else} expressions could be replaced by a single \texttt{switch} statement.}. This could help a bit with performance.

\paragraph{Formalization}
% formal proof of correctness of relaxed operations
% formal proof of correctness of canonicalisation
There is still the need for formal proof on some operations. Mainly for the optimized implementations of the relaxed operations. The canonicalization operation of the vector also needs a formal proof and may be a interesting study case for immutable data structures with simple internal mutation schemes.

%\paragraph{String concatenation} 
%Explore the possibility of improving string concatenation using RRB vector. It would probably not be a good idea to use a vector of characters as a string due to overhead on memory an access time due to boxing of characters (this is speculation). Rather try to represent the result of string concatenations in a vector of strings. If a wrapper is used, it could use the same technique as the canonicalization to generate the actual string if needed and drop the vector. Compare performance with other string representation (like ropes \cite{SPE:SPE4380251203}).

