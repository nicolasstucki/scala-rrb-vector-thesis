I% Chapter Template

\chapter{Related and Future Work} % Main chapter title

\label{RelatedWork} % Change X to a consecutive number; for referencing this chapter elsewhere, use \ref{ChapterX}

\lhead{Related Work} % Change X to a consecutive number; this is for the header on each page - perhaps a shortened title

%----------------------------------------------------------------------------------------
%	SECTION Related Work
%----------------------------------------------------------------------------------------
\section{Related Work}

List of related subjects:
\begin{itemize}
  \item Scala Collections and Parallel Collection
  \item Vector
  \item RRB-Trees and RRB-Vectors
  \item Semi-mutable data structures
  \item Performance and Code specialization (manual staging)
  \item JVM: Arrays, GC, JIT compiler
  \item ScalaMeter
  \item Scala Test
  \item Scala Reflection and Quasicuotes
\end{itemize}

\paragraph{RRB-Vectors in Clojure}

% Describe the improvements proposed with transience

% Describe fundamental difference between this transience and the display one.

\color{red} TODO \color{black}



%----------------------------------------------------------------------------------------
%	SECTION Related Work
%----------------------------------------------------------------------------------------
\section{Future Work}

% find a good measurement of unbalancess to characterise vectors

% characterize vectors on real world programs

% formal proof of correctness of relaxed operations
% formal proof of correctness of canonicalisation

% use macros to define core operations of the vector to simplify expanded code (half of the work is already done on the generators)




