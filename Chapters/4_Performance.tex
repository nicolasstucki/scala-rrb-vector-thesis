I% Chapter Template

\chapter{Performance} % Main chapter title

\label{Performance} % Change X to a consecutive number; for referencing this chapter elsewhere, use \ref{ChapterX}

\lhead{Performance. \emph{Performance in practice and Benchmarks}} % Change X to a consecutive number; this is for the header on each page - perhaps a shortened title


%----------------------------------------------------------------------------------------
%	SECTION In practice: Running on JVM
%----------------------------------------------------------------------------------------
\section{In practice: Running on JVM}
% code interpreted
% code compiled (inlined if hot)


%-----------------------------------
%	SUBSECTION Cost of Abstraction
%-----------------------------------
\subsection{Cost of Abstraction and JIT Inline}



%----------------------------------------------------------------------------------------
%	SECTION Scalameter
%----------------------------------------------------------------------------------------
\section{Measuring performance}
% Scalameter
% JIT inline


%----------------------------------------------------------------------------------------
%	SECTION Generators
%----------------------------------------------------------------------------------------
\section{Generators}
% block sizes
% concat implementation
% balanced, a bit unbalanced and extremely unbalanced



%----------------------------------------------------------------------------------------
%	SECTION Benchmarks
%----------------------------------------------------------------------------------------
\section{Benchmarks}


%-----------------------------------
%	SUBSECTION Apply
%-----------------------------------
\subsection{Apply}

\begin{figure}[h!]
  \centering
  \includegraphics[width=\textwidth]{Benchmarks/Apply_2.pdf}
  \includegraphics[width=\textwidth]{Benchmarks/Apply_3.pdf}
  \label{ApplyBenchmarks}
  \caption{Time to execute 10k apply operations on sequential indices.}
\end{figure}

\begin{figure}[h!]
  \centering
  \includegraphics[width=\textwidth]{Benchmarks/Apply_random_3.pdf}
  \label{ApplyRandomBenchmarks}
  \caption{Time to execute 10k apply operations on random indices.}
\end{figure}

\begin{figure}[h!]
  \centering
  \includegraphics[width=\textwidth]{Benchmarks/apply_blocks.pdf}
  \label{ApplyBlocksBenchmarks}
  \caption{Time to execute 10k apply operations on sequential indices. Comparing performances for different block sizes and different implementation of the concatenation inner branch rebalancing (Copmlete/Quick).}
\end{figure}

%-----------------------------------
%	SUBSECTION Concatenation
%-----------------------------------
\subsection{Concatenation}

\begin{figure}[h!]
  \centering
  \includegraphics[width=\textwidth]{Benchmarks/Concat.png}
  \label{ConcatBenchmarks}
  \caption{Execution time for a concatenation operation on two vectors. In theory (and in practice) Vector conatenation is $O(left + right)$ and the rrbVector concatenation operation is $O(log_{32}(left + right))$.}
\end{figure}

%-----------------------------------
%	SUBSECTION Append
%-----------------------------------
\subsection{Append}

\begin{figure}[h!]
  \centering
  \includegraphics[width=\textwidth]{Benchmarks/Append_2.pdf}
  \includegraphics[width=\textwidth]{Benchmarks/Append_3.pdf}
  \label{AppendBenchmarks}
  \caption{Time to execute 256 append operations. This shows the amortized cost of the append operation.}
\end{figure}


\begin{figure}[h!]
  \centering
  \includegraphics[width=\textwidth]{Benchmarks/Append_blocks_3.pdf}
  \label{IterationBlocksBenchmarks}
  \caption{Time to execute 256 append operations. This shows the amortized cost of the append operation. Comparing performances for different block sizes and different implementation of the concatenation inner branch rebalancing (Copmlete/Quick).}
\end{figure}

%-----------------------------------
%	SUBSECTION Prepend
%-----------------------------------
\subsection{Prepend}

\begin{figure}[h!]
  \centering
  \includegraphics[width=\textwidth]{Benchmarks/Prepend_2.pdf}
  \includegraphics[width=\textwidth]{Benchmarks/Prepend_3.pdf}
  \label{PrependBenchmarks}
  \caption{Time to execute 256 prepend operations. This shows the amortized cost of the prepend operation.}
\end{figure}

%-----------------------------------
%	SUBSECTION Iterator
%-----------------------------------
\subsection{Iterator}

\begin{figure}[h!]
  \centering
  \includegraphics[width=\textwidth]{Benchmarks/Iteration_3.pdf}
  \includegraphics[width=\textwidth]{Benchmarks/Iteration_4.pdf}
  \label{IterationBenchmarks}
  \caption{Excecution time to iterate through all the elements of the vector.}
\end{figure}

\begin{figure}[h!]
  \centering
  \label{IterationBlocksBenchmarks}
  \includegraphics[width=\textwidth]{Benchmarks/Iteration_blocks_3.pdf}
  \caption{Excecution time to iterate through all the elements of the vector. Comparing performances for different block sizes and different implementation of the concatenation inner branch rebalancing (Copmlete/Quick).}
\end{figure}

%-----------------------------------
%	SUBSECTION Builder
%-----------------------------------
\subsection{Builder}

\begin{figure}[h!]
  \centering
  \includegraphics[width=\textwidth]{Benchmarks/Builder_3.pdf}
  \includegraphics[width=\textwidth]{Benchmarks/Builder_4.pdf}
  \label{BuilderBenchmarks}
  \caption{Execution time to build a vector of a given size.}
\end{figure}

%-----------------------------------
%	SUBSECTION Parallel split-combine
%-----------------------------------
\subsection{Parallel split-combine}

\begin{figure}[h!]
  \centering
  \includegraphics[width=\textwidth]{Benchmarks/Parmap_balanced.pdf}
  \label{ParallelBenchmarks}
  \caption{Benchmark on map and parallel map using the function (\textsc{x=>x}) to show the difference time used in the framework. This time represents the time spent in the splitters and combiners of the parallel collection (iterator and builder for the sequential version).}
\end{figure}

\begin{figure}[h!]
  \centering
  \includegraphics[width=\textwidth]{Benchmarks/parmap_unbalanced.pdf}
  \label{ParallelUnbalancedBenchmarks}
  \caption{Benchmark on map and parallel map using the function (\textsc{x=>x}) to show the difference time used in the framework. This time represents the time spent in the splitters and combiners of the parallel collection.}
\end{figure}



